\label{sec:intro}

Целью данного дипломного проекта является разработка ПС, способного с высокой степенью точности определять вероятные места скопления людей в помещениях.
Очевидным способом выполнения этой функции является моделирование пешеходных потоков.

Моделирование пешеходных потоков "--- достаточно новая область в моделировании. Она выделилась из моделирования транспортных потоков в 1980-х годах.
Область применения моделей пешеходных потоков – использование для тестирования различных сооружений, обслуживающих интенсивные потоки людей.
Примерами таких сооружений могут служить вокзалы, станции метро, торговые центры, стадионы. Их тестирование на этапе проектирования позволяет выявить слабые места и устранить их путем перепланировки.

Было разработано множество различных моделей пешеходных потоков, каждая из которых имеет свои недостатки и преимущества.
Более подробно модели пешеходных потоков, а также предложенные модификации выбранной модели рассмотрены в первом и втором разделах дипломного проекта.

В третьем разделе описан процесс проектирования ПС, а именно описаны принципы и технологии, используемые для проектирования ПС, архитектура ПС и форматы данных, а также более детальное описание модулей ПС.

Четвертый раздел посвящен тестированию ПС. Данный раздел содержит описание способов тестирования разрабатываемого ПС.

В пятом разделе под названием <<Методика использования разработанного программного средства>> приведено подробное иллюстрированное руководство пользователя к разрабатываемому ПС. Данное руководство пользователя включает в себя подробное описание входного формата конфигурации симуляции и различных режимов работы ПС.

В разделе <<Технико-экономическое обоснования>> произведен рассчет экономической эффективности разрабатываемого ПС.

В разделе <<Обеспечение безопасности при разработке и испытании информационной системы>> рассмотрены вопросы, связанные с охраной труда в процессе разработки ПС.
