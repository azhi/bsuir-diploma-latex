\sectioncentered*{Краткое введение}

Темой данной работы является корректировка разработанной в рамках дипломного проекта имитационной модели пешеходных потоков для работы в условиях массовой паники.

Упомянутый дипломный проект ставил своей целью разработку программного средства, способного предсказывать вероятные места скопления людей в помещениях.
Предполагаемые задачи, решаемые данным программным средством, включали в себя задачу по оценке безопасности исследуемого сооружения.
Данная работа значительно расширяет возможности разработанной имитационной модели и программного средства, позволяя сконцентрироваться на исследовании таких важных параметров сооружения, как время эвакуации.

Однако сценарий эвакуации отличается от нормального функционирования сооружения "--- при эвакуации возможно возникновение массовой паники.
Именно поэтому данная работа ставит своей целью модификацию модели для работы в условиях массовой паники.
С помощью модифицированного программного средства можно будет оценить время эвакуации из заданного сооружения при различных "уровнях" паники.

Работа включает в себя подробный обзор существующей научной литературы на обсуждаемую тему, описание разработанной модели и ее характеристик,
описание процесса проектирования модификаций программного средства, измененное руководство пользователя к разрабатываемому программному средству, оценку результатов, полученных с помощью программного средства,
и заключение.

\sectioncentered*{Общая характеристика работы}
\addcontentsline{toc}{section}{Общая характеристика работы}

\label{sec:general_characteristics}

\subsection*{\textbf{Цель и задачи исследования}}

Цель магистерской диссертации "--- провести исследование влияния массовой паники на поведение людей в общем
и на характеристики пешеходных потоков в частности, а также оценить степень и механизмы влияния массовой паники
на время эвакуации.

Для достижения поставленной цели необходимо решить следующие задачи:

\begin{itemize}
  \item Провести исследование существующих моделей массовой паники: выделить основные атрибуты моделей, оценить их результаты.
  \item На основе исследования существующих моделей разработать собственную модель, учитывающую некоторые дополнительные аспекты рассматриваемой области.
  \item Разработать программное средство, использующее модель массовой паники собственной разработки.
  \item Провести эксперименты, позволяющие оценить влияние массовой паники на исследуемые характеристики.
\end{itemize}

Объектом магистерской диссертации является явление массовая паники.

Предметом магистерской диссертации является влияние массовой паники на характеристики пешеходных потоков и на время эвакуации.
Основная гипотеза, положенная в основу работы: массовая паника негативно сказывается на времени эвакуации.

\subsection*{\textbf{Личный вклад соискателя}}

Результаты, приведенные в диссертации, получены соискателем лично.
Вклад научного руководителя \mastersCharSupervisor, заключается в формулировке цели и задач исследования.

\subsection*{\textbf{Опубликованность результатов диссертации}}

По теме диссертации опубликована 1 печатная работа в сборниках трудов и материалов международных конференций.

\subsection*{\textbf{Структура и объем диссертации}}

Диссертация состоит из введения, общей характеристики работы, пяти глав, заключения, библиографического списка и одного приложения.
В главе 1 приводится обзор существующей научной литературы на обсуждаемую тему. Глава 2 посвящена описанию разработанной модели массовой паники.
Глава 3 включает в себя описание модификаций, выполненных на программном средстве, позволяющих ему использовать разработанную в главе 2 модель.
В главе 4 приводится краткое измененное руководство пользователя к разработанному программному средству.
Глава 5 описывает проведенные эксперименты и анализирует их результаты.

Общий объем работы составляет 75 страниц, из которых основного текста "--- 75 страниц, 17 рисунков на 13 страницах, 1 листинг исходных кодов на 20 страницах,
список использованных источников из 32 наименований на 3 страницах.


\sectioncentered*{Основное содержание}

Во введении определена область и указаны основные направления исследования.

В первой главе произведен анализ существующей научной литературы на тему моделирования массово паники.
Были рассмотрены работы таких авторов как Д. Хелбинг, П. Молнар, А. Кирчнер, Ч. Рен, Дж. Патрикса и др.
Рассмотрены различные подходы к моделированию пешеходных потоков в общем и к моделированию пешеходных
потоков в сценарии паники в частности. Рассмотренные подходы включают в себя газокинетическую модель,
модель социальных сил, модель на основе клеточных автоматов, агентную модель и расчетную модель.

Во второй главе описана разработка собственной модели паники на основе анализа существующих моделей.
Приведены требования к разрабатываемой модели, а затем по каждому требованию составлен компонент модели,
учитывающий данное требование. Список требований, реализованные в модели, представлены ниже:

\begin{itemize}
  \item люди становятся более нервными, то есть быстрее и чаще принимают необоснованные и иррациональные решения;
  \item люди стараются двигаться значительно быстрее чем обычно;
  \item люди начинают толкаться, взаимодействия между людьми становятся физическими по природе;
  \item люди демонстрируют <<стадное поведение>>, то есть делают то же, что и другие люди вокруг них.
\end{itemize}

Также глава включает в себя конкретное математическое описание каждого компонента модели, призванного учесть один или несколько аспектов из требований:
\begin{itemize}
  \item сила флуктуации "--- учитывает требование о необоснованных и иррациональных решениях;
  \item сила физического отталкивания "--- учитывает требование об учете физических взаимодействий;
  \item сила <<стадного поведения>> "--- учитывает требование об учете желания делать то же, что и другие люди вокруг них;
  \item уровни паники и их распространение "--- уникальный компонент разработанной модели, учитывает распространение паники.
\end{itemize}


В третьей главе описан процесс модификации существующего программного средства для работы с разработанной
во второй главе моделью. Представлен план изменений, которые будет необходимо внести в программное средство,
а также решения, принятые в процессе модификаций. В конце раздела приводится анализ архитектуры разработанного
программного средства, в частности рассматривается вопрос сложности внесения изменений.

В четвертой главе представлено измененное руководство пользователя к разработанному программному средству.
Так как интерфейс программного средства не претерпевал сильных изменений, данная глава получилось достаточно короткой.

В пятой главе представлены описания экспериментов, поставленных с использованием разработанного программного средства,
а также анализ полученных результатов. Результаты, полученные с помощью разработанного комплекса модели массовой паники и
программного средства, можно поделить на две группы.

В первую группу входят негативные результаты, не отражающие
никаких зависимостей между конкретным аспектом поведения людей и исследуемыми характеристиками.
К данной группе можно отнести результаты по учету физических контактов между людьми и по учету
желания людей придерживаться того же направления движения что и люди вокруг них.
Представлен анализ причин таких результатов, а так же предложены будущие улучшения модели и программного средства
для исправления ситуации.

Во вторую группу входят положительные результаты, выявившие определенные закономерности
в исследуемых характеристиках. К ним можно отнести результаты по учету случайных иррациональных решений
людей, правила распространения паники и результаты общего исследования о влиянии паники на время эвакуации.
В этих результатах выявлена положительная зависимость между аспектом поведения людей о временем эвакуации,
которую можно сформулировать как <<Увеличение уровня паники приводит к замедлению эвакуации>>,
что подтверждает основную гипотезу исследования.

В целом, разработанный комплекс модели массовой паники и программного средства, ее реализующего,
демонстрирует некоторые интересные результаты, однако в некоторых аспектах не полностью отражает реальность.
В заключении предложены направления для дальнейшей работы, которые позволят улучшить модель и программное средство
для возможности проведения более объемлющего исследования представленной темы.

\sectioncentered*{Заключение}

\subsection*{\textbf{Основные научные результаты диссертации}}

\begin{itemize}
  \item Проведено исследование существующих моделей массовой паники на основе различных подходов;
  \item Предложена новая модель массовой паники на основе существующих, но со значительными улучшениями;
  \item Произведена оценка влияния каждого аспекта разработанной модели массовой паники;
  \item Составлены предложения по дальнейшему улучшению разработанной модели массовой паники.
\end{itemize}


\renewcommand{\bibsection}{\sectioncentered*{Список опубликованных работ}}
\addcontentsline{toc}{section}{Cписок опубликованных работ}

\nocite{mypub}
\bibliographystyle{styles/belarus-specific-utf8gost780u}
\bibliography{bibliography_database}
