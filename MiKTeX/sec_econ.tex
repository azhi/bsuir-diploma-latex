\newcommand{\byr}{Br}

\section{Технико-экономическое обоснование}

% Begin Calculations

\FPeval{\totalProgramSize}{7110}
\FPeval{\totalProgramSizeCorrected}{6000}

\FPeval{\normativeManDays}{162}

\FPeval{\additionalComplexity}{0.4}
\FPeval{\complexityFactor}{clip(1 + \additionalComplexity)}

\FPeval{\stdModuleUsageFactor}{0.8}
\FPeval{\originalityFactor}{0.7}

\FPeval{\adjustedManDaysExact}{clip( \normativeManDays * \complexityFactor * \stdModuleUsageFactor * \originalityFactor )}
\FPround{\adjustedManDays}{\adjustedManDaysExact}{0}

\FPeval{\daysInYear}{365}
\FPeval{\redLettersDaysInYear}{6}
\FPeval{\weekendDaysInYear}{104}
\FPeval{\vocationDaysInYear}{21}
\FPeval{\workingDaysInYear}{ clip( \daysInYear - \redLettersDaysInYear - \weekendDaysInYear - \vocationDaysInYear ) }

\FPeval{\developmentTimeMonths}{3}
\FPeval{\developmentTimeYearsExact}{clip(\developmentTimeMonths / 12)}
\FPround{\developmentTimeYears}{\developmentTimeYearsExact}{2}
\FPeval{\requiredNumberOfProgrammersExact}{ clip( \adjustedManDays / (\developmentTimeYears * \workingDaysInYear) ) }

% тут должно получаться 2 ))
\FPround{\requiredNumberOfProgrammers}{\requiredNumberOfProgrammersExact}{0}

\FPeval{\tariffRateFirst}{500000}
\FPeval{\tariffFactorFst}{3.04}
\FPeval{\tariffFactorSnd}{3.48}


\FPeval{\employmentFstExact}{clip( \adjustedManDays / \requiredNumberOfProgrammers )}
\FPtrunc{\employmentFst}{\employmentFstExact}{0}

\FPeval{\employmentSnd}{clip(\adjustedManDays - \employmentFst)}


\FPeval{\workingHoursInMonth}{160}
\FPeval{\salaryPerHourFstExact}{clip( \tariffRateFirst * \tariffFactorFst / \workingHoursInMonth )}
\FPeval{\salaryPerHourSndExact}{clip( \tariffRateFirst * \tariffFactorSnd / \workingHoursInMonth )}
\FPround{\salaryPerHourFst}{\salaryPerHourFstExact}{0}
\FPround{\salaryPerHourSnd}{\salaryPerHourSndExact}{0}

\FPeval{\bonusRate}{1.5}
\FPeval{\workingHoursInDay}{8}
\FPeval{\totalSalaryExact}{clip( \workingHoursInDay * \bonusRate * ( \salaryPerHourFst * \employmentFst + \salaryPerHourSnd * \employmentSnd ) )}
\FPround{\totalSalary}{\totalSalaryExact}{0}

\FPeval{\additionalSalaryNormative}{20}

\FPeval{\additionalSalaryExact}{clip( \totalSalary * \additionalSalaryNormative / 100 )}
\FPround{\additionalSalary}{\additionalSalaryExact}{0}

\FPeval{\socialNeedsNormative}{0.6}
\FPeval{\socialProtectionNormative}{34}
\FPeval{\socialProtectionFund}{ clip(\socialNeedsNormative + \socialProtectionNormative) }

\FPeval{\socialProtectionCostExact}{clip( (\totalSalary + \additionalSalary) * \socialProtectionFund / 100 )}
\FPround{\socialProtectionCost}{\socialProtectionCostExact}{0}

\FPeval{\taxWorkProtNormative}{4}
\FPeval{\taxWorkProtCostExact}{clip( (\totalSalary + \additionalSalary) * \taxWorkProtNormative / 100 )}
\FPround{\taxWorkProtCost}{\taxWorkProtCostExact}{0}
\FPeval{\taxWorkProtCost}{0} % это считать не нужно, зануляем чтобы не менять формулы

\FPeval{\stuffOnHundredLoC}{1.27}
\FPeval{\stuffCostExact}{clip( \stuffOnHundredLoC * \totalProgramSizeCorrected / 100 )}
\FPeval{\stuffCost}{\stuffCostExact}

\FPeval{\timeToDebugCodeNormative}{27}
\FPeval{\reducingTimeToDebugFactor}{0.3}
\FPeval{\adjustedTimeToDebugCodeNormative}{ clip( \timeToDebugCodeNormative * \reducingTimeToDebugFactor ) }

\FPeval{\oneHourMachineTimeCost}{3000}

\FPeval{\machineTimeCostExact}{ clip( \oneHourMachineTimeCost * \totalProgramSizeCorrected / 100 * \adjustedTimeToDebugCodeNormative ) }
\FPround{\machineTimeCost}{\machineTimeCostExact}{0}

\FPeval{\businessTripNormative}{30}
\FPeval{\businessTripCostExact}{ clip( \totalSalary * \businessTripNormative / 100 ) }
\FPround{\businessTripCost}{\businessTripCostExact}{0}

\FPeval{\otherCostNormative}{20}
\FPeval{\otherCostExact}{clip( \totalSalary * \otherCostNormative / 100 )}
\FPround{\otherCost}{\otherCostExact}{0}

\FPeval{\overheadCostNormative}{100}
\FPeval{\overallCostExact}{clip( \totalSalary * \overheadCostNormative / 100 )}
\FPround{\overheadCost}{\overallCostExact}{0}

\FPeval{\overallCost}{clip( \totalSalary + \additionalSalary + \socialProtectionCost + \taxWorkProtCost + \stuffCost + \machineTimeCost + \businessTripCost + \otherCost + \overheadCost ) }

\FPeval{\supportNormative}{30}
\FPeval{\softwareSupportCostExact}{clip( \overallCost * \supportNormative / 100 )}
\FPround{\softwareSupportCost}{\softwareSupportCostExact}{0}


% \FPeval{\baseCost}{ clip( \overallCost + \softwareSupportCost ) }
\FPeval{\baseCost}{ clip( \overallCost  ) }

\FPeval{\profitability}{35}
\FPeval{\incomeExact}{clip( \baseCost / 100 * \profitability )}
\FPround{\income}{\incomeExact}{0}

\FPeval{\estimatedPrice}{clip( \income + \baseCost )}

\FPeval{\localRepubTaxNormative}{3.9}
\FPeval{\localRepubTaxExact}{clip( \estimatedPrice * \localRepubTaxNormative / (100 - \localRepubTaxNormative) )}
\FPround{\localRepubTax}{\localRepubTaxExact}{0}
\FPeval{\localRebubTax}{0} % это считать не нужно, зануляем чтобы не менять формулы

\FPeval{\ndsNormative}{20}
\FPeval{\ndsExact}{clip( (\estimatedPrice + \localRepubTax) / 100 * \ndsNormative )}
\FPround{\nds}{\ndsExact}{0}


\FPeval{\sellingPrice}{clip( \estimatedPrice + \localRepubTax + \nds )}

\FPeval{\taxForIncome}{18}
\FPeval{\incomeWithTaxes}{clip(\income * (1 - \taxForIncome / 100))}
\FPround\incomeWithTaxes{\incomeWithTaxes}{0}

% End Calculations

\subsection{Введение и исходные данные}

Целью дипломного проекта является создание ПС определения вероятных мест скопления людей в помещениях,
направленного на решение актуальных задач, возникающих при разработке архитектурных проектов.
Разрабатываемое ПС позволяет решать задачи обеспечения безопасности, оптимизации путей эвакуации,
максимизация прибыли путем определения наиболее привлекательных мест установки коммерческих объектов и многие другие.
В данном разделе определяется экономическая эффективность данного ПС.
Экономическая эффективность выступает в виде прибыли для разработчика и экономии затрат для пользователя.

Исходные данные для разрабатываемого проекта указаны в таблице~\ref{table:econ:initial_data}.

\begin{table}[!ht]
\caption{Исходные данные}
\label{table:econ:initial_data}
  \centering
  \begin{tabular}{| >{\raggedright}m{0.54\textwidth} 
                  | >{\centering}m{0.11\textwidth} 
                  | >{\centering}m{0.14\textwidth} 
                  | >{\centering\arraybackslash}m{0.11\textwidth}|}
    \hline
    {\begin{center}
      Наименование
    \end{center} } & Условное обозначение & Единицы измерения & Значение \\
    % \hline Категория сложности & & единиц & 2 \\
    % \hline Коэффициент сложности & $\text{К}_\text{с}$ & единиц & \num{\complexityFactor} \\
    % \hline Степень использования при разработке стандартных модулей & $\text{К}_\text{т}$ & единиц & \num{\stdModuleUsageFactor} \\
    % \hline Коэффициент новизны & $\text{К}_\text{н}$ & единиц & \num{\originalityFactor} \\
    % \hline Норматив расходов на сопровождение и адаптацию ПО & $\text{Н}_\text{рса}$ & $\%$ & \num{\supportNormative} \\
    \hline Срок разработки проекта & $\text{Т}_\text{р}$ & месяцев & \num{\developmentTimeMonths} \\
    \hline Продолжительность рабочего дня & $\text{Т}_\text{ч}$ & часов & \num{\workingHoursInDay} \\
    \hline Месячная тарифная ставка первого разряда & $\text{Т}_{\text{м}_1}$ & \byr{} & \num{\tariffRateFirst} \\
    \hline Коэффициент премирования & $\text{К}_\text{п}$ & единиц & \num{\bonusRate} \\
    \hline Норматив дополнительной заработной платы & $\text{Н}_\text{д}$ & $\%$ & \num{\additionalSalaryNormative} \\
    \hline Норматив отчислений в ФСЗН и обязательное страхование & $\text{Н}_\text{сз}$ & $\%$ & \num{\socialProtectionFund} \\
    \hline Норматив командировочных расходов & $\text{Н}_\text{к}$ & $\%$ & \num{\businessTripNormative} \\
    \hline Норматив прочих затрат & $\text{Н}_\text{пз}$ & $\%$ & \num{\otherCostNormative} \\

    \hline Норматив накладных расходов & $ \text{Н}_\text{рн}$ & $\%$ & \num{\overheadCostNormative} \\
    \hline Прогнозируемый уровень рентабельности & $\text{У}_\text{рп}$ & $\%$ & \num{\profitability} \\
    \hline Норматив НДС & $\text{Н}_\text{дс}$ & $\%$ & \num{\ndsNormative} \\
    \hline Норматив налога на прибыль & $\text{Н}_\text{п}$ & $\%$ & \num{\taxForIncome} \\
    \hline Цена одного часа машинного времени & $\text{Н}_\text{мв }$ & \byr{} & \num{\oneHourMachineTimeCost} \\
    \hline
  \end{tabular}
\end{table}

\subsection{Расчёт сметы затрат и цены}

Оценка стоимости ПО у разработчика предполагает составление сметы затрат, которая включает следующие статьи расходов:
\begin{itemize}
  \item заработную плату исполнителей, основную ($\text{З}_\text{o}$) и дополнительную ($\text{З}_\text{д}$);
  \item отчисления в фонд социальной защиты населения ($\text{З}_\text{сз}$);
  \item налоги от фонда оплаты труда ($\text{Н}_\text{е}$);
  \item материалы и комплектующие ($\text{М}$);
  \item машинное время ($\text{Р}_\text{м}$);
  \item расходы на научные командировки ($\text{Р}_\text{нк}$);
  \item прочие прямые расходы ($\text{П}_\text{з}$);
  \item накладные расходы ($\text{Р}_\text{н}$).
\end{itemize}

На основании общей суммы расходов по всем статьям и результатов исследования рынка ПО определяется плановая отпускаемая цена ($\text{Ц}_\text{о}$) с учетом прибыли (рентабельности) и налогов.

Для составления сметы затрат на создание ПО необходима предварительная оценка трудоемкости ПО, в частности его объёма.
Расчет объёма программного продукта (количества LoC - Line of Code, строк исходного кода) предполагает определение типа программного обеспечения, техническое обоснование функций ПО и определение объёма каждой функций.
Согласно классификации типов программного обеспечения~\cite[с.~59,~приложение 1]{econ_palicyn}, разрабатываемое ПС можно классифицировать как ПО методo"=ориентированных расчетов.

Общий объём программного продукта определяется исходя из количества и объёма функций, реализованных в программе, определяется как:
\begin{equation}
  \label{eq:econ:total_program_size}
  V_{o} = \sum_{i = 1}^{n} V_{i}
\end{equation}
\begin{explanation}
где & $ V_{i} $ & объём отдельной функции ПО, LoC; \\
    & $ n $ & общее число функций.
\end{explanation}

На стадии технико-экономического обоснования проекта рассчитать точный объём функций невозможно.
Вместо вычисления точного объёма функций применяются приблизительные оценки на основе данных по аналогичным проектам или по нормативам~\cite[с.~61,~приложение 2]{econ_palicyn}, которые приняты в организации.

Каталог аналогов программного обеспечения предназначен для предварительной оценки объёма ПО методом структурной аналогии.
В разных организациях в зависимости от технических и организационных условий, в которых разрабатывается ПО, предварительные оценки могут корректироваться на основе экспертных оценок.
Уточненный объём ПО рассчитывается по формуле:
\begin{equation}
  \label{eq:econ:total_program_size_corrected}
  V_{\text{у}} = \sum_{i = 1}^{n} V_{i}^{\text{у}} \text{\,,}
\end{equation}
\begin{explanation}
где & $ V_{i}^{\text{y}} $ & уточненный объём отдельной функции ПО, LoC; \\
    & $ n $ & общее число функций.
\end{explanation}

Перечень и объём функций программного модуля перечислен в таблице~\ref{table:econ:function_sizes}.

\begin{table}[ht]
\caption{Перечень и объём функций программного модуля}
\label{table:econ:function_sizes}
\centering
  \begin{tabular}{| >{\centering}m{0.12\textwidth} 
                  | >{\raggedright}m{0.40\textwidth} 
                  | >{\centering}m{0.18\textwidth} 
                  | >{\centering\arraybackslash}m{0.18\textwidth}|}

  \hline
         \multirow{2}{0.12\textwidth}[-0.5em]{\centering \No{} функции}
       & \multirow{2}{0.40\textwidth}[-0.55em]{\centering Наименование (содержание)} 
       & \multicolumn{2}{c|}{\centering Объём функции, LoC} \tabularnewline

  \cline{3-4} & 
       & { по каталогу ($ V_{i} $) }
       & { уточненный ($ V_{i}^{\text{у}} $) } \tabularnewline

  \hline 102 & Контроль, предварительная обработка и ввод информации & \num{520} & \num{520} \tabularnewline
  \hline 103 & Анализ входного языка (синтаксический и семантический) & \num{630} & \num{630} \tabularnewline
  \hline 111 & Управление вводом/выводом & \num{2700} & \num{1500} \tabularnewline

  \hline 301 & Формирование последовательного файла & \num{340} & \num{340} \tabularnewline
  \hline 305 & Обработка файлов & \num{750} & \num{750} \tabularnewline

  \hline 405 & Система настройки ПО & \num{250} & \num{500} \tabularnewline

  \hline 506 & Обработка ошибочных и сбойных ситуаций & \num{430} & \num{430} \tabularnewline
  \hline 507 & Обеспечение интерфейса между компонентами & \num{730} & \num{730} \tabularnewline

  \hline 605 & Вспомогательные и сервисные программы & \num{460} & \num{300} \tabularnewline 

  \hline 707 & Графический вывод результатов & \num{300} & \num{300} \tabularnewline

  \hline

  Итог & & {\num{\totalProgramSize}} & {\num{\totalProgramSizeCorrected}} \tabularnewline

  \hline

  \end{tabular}
\end{table}

По приведенным данным уточненный объём некоторых функций изменился, и общий объём ПО составил $ V_{o} = \SI{\totalProgramSize}{\text{LoC}} $, а общий уточненный объём ПО~"---~$ V_{\text{у}} = \SI{\totalProgramSizeCorrected}{\text{LoC}} $.

По уточненному объёму ПО и нормативам затрат труда в расчете на единицу объёма определяются нормативная и общая трудоемкость разработки ПО.
ПО относится ко второй категории сложности: предполагается его использование для реализации особо сложных инженерных и научных расчетов, также необходимо обеспечить настройку ПО на изменения структур входных и выходных данных~\cite[с.\,66, приложение~4, таблица~П.4.1]{econ_palicyn}.
По полученным данным определяется нормативная трудоемкость разработки ПО.
Согласно укрупненным нормам времени на разработку ПО в зависимости от уточненного объёма ПО и группы сложности ПО~\cite[c.~64,~приложение~3]{econ_palicyn} нормативная трудоемкость разрабатываемого проекта составляет~$ \text{Т}_\text{н} = \SI{\normativeManDays}{\text{чел.} / \text{дн.}}  $

Нормативная трудоемкость служит основой для оценки общей трудоемкости~$ \text{Т}_\text{о} $, расчет которой осуществляется различными способами в зависимости от размеров проекта.
Используем формулу (\ref{eq:econ:effort_common}) для оценки общей трудоемкости для небольших проектов:
\begin{equation}
  \label{eq:econ:effort_common}
  \text{Т}_\text{о} = \text{Т}_\text{н} \cdot 
                      \text{К}_\text{с} \cdot 
                      \text{К}_\text{т} \cdot 
                      \text{К}_\text{н}
\end{equation}
\begin{explanation}
где & $ \text{К}_\text{с} $ & коэффициент, учитывающий сложность ПО; \\
    & $ \text{К}_\text{т} $ & поправочный коэффициент, учитывающий степень использования при разработке стандартных модулей; \\
    & $ \text{К}_\text{н} $ & коэффициент, учитывающий степень новизны ПО.
\end{explanation}

Дополнительные затраты труда на разработку ПО учитываются через коэффициент сложности, который вычисляется по формуле
\begin{equation}
\label{eq:econ:complexity_coeff}
  \text{К}_{\text{с}} = 1 + \sum_{i = 1}^n \text{К}_{i}
\end{equation}
\begin{explanation}
где & $ \text{К}_{i} $ & коэффициент, соответствующий степени повышения сложности ПО за счет конкретной характеристики; \\
    & $ n $ & количество учитываемых характеристик.
\end{explanation}

Применительно к разрабатываемому ПС учитываются следующие дополнительные коэффициенты сложности ПО~\cite[c.~66, приложение~4, таблица~П.4.2]{econ_palicyn}:
\begin{itemize}
  \item функционирование ПО в расширенной операционной среде (связь с другими по) "--- $0,08$;
  \item интерактивный доступ "--- $0,06$;
  \item наличие у ПО одновременно множества характеристик из признаков категории сложности (
    режим работы в реальном времени,
    машинная графика,
    обеспечение настройки ПО на изменения структур входных и выходных данных,
    реализация особо сложных инженерных и научных расчетов
    ) "--- $0,26$.
\end{itemize}

Таким образом, коэффициент сложности определяется как:
\begin{equation}
\label{eq:econ:complexity_coeff_calc}
  \text{К}_{\text{с}} = \num{1} + \num{\additionalComplexity} = \num{\complexityFactor}
\end{equation}

Разрабатываемое ПО использует стандартные компоненты (библиотеки, пакеты). Степень использования стандартных компонентов определяется коэффициентом использования стандартных модулей~"---~$ \text{К}_\text{т} $.
Согласно справочным данным~\cite[c.~68,~приложение~4, таблица~П.4.5]{econ_palicyn} указанный коэффициент для разрабатываемого приложения $ \text{К}_\text{т} = \num{\stdModuleUsageFactor} $.

Трудоемкость создания ПО также зависит от его новизны и наличия аналогов.
Разрабатываемое ПО не является новым, существуют аналогичные разработки у различных компаний.
Влияние степени новизны на трудоемкость создания ПО определяется коэффициентом новизны~"---~$ \text{К}_\text{н} $.
Согласно справочным данным~\cite[c.~67, приложение~4, таблица~П.4.4]{econ_palicyn} для разрабатываемого ПО $ \text{К}_\text{н} = \num{\originalityFactor} $.
Подставив приведенные выше коэффициенты для разрабатываемого ПО в формулу~(\ref{eq:econ:effort_common}) получим общую трудоемкость разработки
\begin{equation}
  \label{eq:econ:effort_common_calc}
  \text{Т}_\text{о} = \num{\normativeManDays} \times \num{\complexityFactor} \times \num{\stdModuleUsageFactor} \times \num{\originalityFactor} \approx \SI{\adjustedManDays}{\text{чел.}/\text{дн.}}
\end{equation}

На основе общей трудоемкости и требуемых сроков реализации проекта вычисляется плановое количество разработчиков и плановые сроки, необходимые для реализации проекта в целом.
Численность исполнителей проекта рассчитывается по формуле:
\begin{equation}
  \label{eq:econ:num_of_programmers}
  \text{Ч}_\text{р} = \frac{\text{Т}_\text{о}}{\text{Т}_\text{р} \cdot \text{Ф}_\text{эф}}
\end{equation}
\begin{explanation}
где & $ \text{Т}_\text{о} $ & общая трудоемкость разработки проекта, $ \text{чел.}/\text{дн.} $; \\
    & $ \text{Ф}_\text{эф} $ & эффективный фонд времени работы одного работника в течение года, дн.; \\
    & $ \text{Т}_\text{р} $ & срок разработки проекта, лет.
\end{explanation}

Эффективный фонд времени работы одного разработчика вычисляется по формуле
\begin{equation}
  \label{eq:econ:effective_time_per_programmer}
  \text{Ф}_\text{эф} = 
    \text{Д}_\text{г} -
    \text{Д}_\text{п} -
    \text{Д}_\text{в} -
    \text{Д}_\text{о}
\end{equation}
\begin{explanation}
где & $ \text{Д}_\text{г} $ & количество дней в году, дн.; \\
    & $ \text{Д}_\text{п} $ & количество праздничных дней в году, не совпадающих с выходными днями, дн.; \\
    & $ \text{Д}_\text{в} $ & количество выходных дней в году, дн.; \\
    & $ \text{Д}_\text{п} $ & количество дней отпуска, дн.
\end{explanation}

Согласно данным, приведенным в производственном календаре для пятидневной рабочей недели в 2015 году для Беларуси~\cite{belcalendar_2015}, фонд рабочего времени составит
\begin{equation}
  \text{Ф}_\text{эф} = \num{\daysInYear} - \num{\redLettersDaysInYear} - \num{\weekendDaysInYear} - \num{\vocationDaysInYear} = \SI{\workingDaysInYear}{\text{дн.}}
\end{equation}

Учитывая срок разработки проекта $ \text{Т}_\text{р} = \SI{\developmentTimeMonths}{\text{мес.}} = \SI{\developmentTimeYears}{\text{года}} $, общую трудоемкость и фонд эффективного времени одного работника, вычисленные ранее, можем рассчитать численность исполнителей проекта
\begin{equation}
  \label{eq:econ:num_of_programmers_calc}
  \text{Ч}_\text{р} = 
    \frac{\num{\adjustedManDays}}
         {\num{\developmentTimeYears} \times \num{\workingDaysInYear}} 
    \approx \SI{\requiredNumberOfProgrammers}{\text{разработчика}}.
\end{equation}

Вычисленные оценки показывают, что для выполнения запланированного проекта в указанные сроки необходимо два рабочих.
В соответствии со штатным расписанием на разработке будут заняты:
\begin{itemize}
  \item программист \Rmnum{1}-категории "--- тарифный разряд "--- $13$; тарифный коэффициент "--- $\num{\tariffFactorFst}$; плановый фонд рабочего времени "--- $\SI{\employmentFst}{\text{дн.}}$;
  \item ведущий программист "--- тарифный разряд "--- $15$; тарифный коэффициент "--- $\num{\tariffFactorSnd}$; плановый фонд рабочего времени "--- $\SI{\employmentSnd}{\text{дн.}}$;
\end{itemize}

Месячная тарифная ставка одного работника вычисляется по формуле
\begin{equation}
  \label{eq:econ:month_salary}
  \text{Т}_\text{ч} = 
    \frac { \text{Т}_{\text{м}_{1}} \cdot \text{Т}_{\text{к}} } 
          { \text{Ф}_{\text{р}} }  \text{\,,}
\end{equation}
\begin{explanation}
где & $ \text{Т}_{\text{м}_{1}} $ & месячная тарифная ставка 1-го разряда, \byr; \\
    & $ \text{Т}_{\text{к}} $ & тарифный коэффициент, соответствующий установленному тарифному разряду; \\
    & $ \text{Ф}_{\text{р}} $ & среднемесячная норма рабочего времени, час.
\end{explanation}

Подставив данные о занятых на проектах разработчиках в формулу~(\ref{eq:econ:month_salary}), приняв значение тарифной ставки 1-го разряда $ \text{Т}_{\text{м}_{1}} = \SI{\tariffRateFirst}{\text{\byr}} $ и среднемесячную норму рабочего времени $ \text{Ф}_{\text{р}} = \SI{\workingHoursInMonth}{\text{часов}} $ получаем
\begin{equation}
  \label{eq:econ:month_salary_calc1}
  \text{Т}_{\text{ч}}^{\text{прогр. \Rmnum{1}-разр.}} = \frac{ \num{\tariffRateFirst} \times \num{\tariffFactorFst} } { \num{\workingHoursInMonth} } = \SI{\salaryPerHourFst}{\text{\byr}/\text{час;}}
\end{equation}
\begin{equation}
  \label{eq:econ:month_salary_calc2}
  \text{Т}_{\text{ч}}^{\text{вед. прогр.}} = \frac{ \num{\tariffRateFirst} \times \num{\tariffFactorSnd} } { \num{\workingHoursInMonth} } = \SI{\salaryPerHourSnd}{\text{\byr}/\text{час.}}
\end{equation}

Основная заработная плата исполнителей рассчитывается по формуле 
\begin{equation}
  \label{eq:econ:total_salary}
  \text{З}_{\text{о}} = \sum^{n}_{i = 1} 
                        \text{Т}_{\text{ч}}^{i} \cdot
                        \text{Т}_{\text{ч}} \cdot
                        \text{Ф}_{\text{п}} \cdot
                        \text{К}
\end{equation}
\begin{explanation}
где & $ \text{Т}_{\text{ч}}^{i} $ & часовая тарифная ставка \mbox{$ i $-го} исполнителя, \byr$/$час; \\
    & $ \text{Т}_{\text{ч}} $ & количество часов работы в день, час; \\
    & $ \text{Ф}_{\text{п}} $ & плановый фонд рабочего времени \mbox{$ i $-го} исполнителя, дн.; \\
    & $ \text{К} $ & коэффициент премирования.
\end{explanation}

Подставив значения в формулу~(\ref{eq:econ:total_salary}) и приняв коэффициент премирования $ \text{К} = \num{\bonusRate} $ получим
\begin{equation}
  \label{eq:econ:total_salary_calc}
  \text{З}_{\text{о}} = (\salaryPerHourFst \times \num{\employmentFst} + \salaryPerHourSnd \times \num{\employmentSnd}) \times \num{\workingHoursInDay} \times \num{\bonusRate} = \SI{\totalSalary}{\text{\byr}}
\end{equation}

Дополнительная заработная плата включает выплаты предусмотренные законодательством о труде и определяется по нормативу в процентах от основной заработной платы
\begin{equation}
  \label{eq:econ:additional_salary}
  \text{З}_{\text{д}} = 
    \frac {\text{З}_{\text{о}} \cdot \text{Н}_{\text{д}}} 
          {100\%}
\end{equation}
\begin{explanation}
  где & $ \text{Н}_{\text{д}} $ & норматив дополнительной заработной платы, $ \% $.
\end{explanation}

Приняв норматив дополнительной заработной платы $ \text{Н}_{\text{д}} = \num{\additionalSalaryNormative\%} $ и подставив известные данные в формулу~(\ref{eq:econ:additional_salary}) получим
\begin{equation}
  \label{eq:econ:additional_salary_calc}
  \text{З}_{\text{д}} = 
    \frac{\num{\totalSalary} \times 20\%}
         {100\%} \approx \SI{\additionalSalary}{\text{\byr}}
\end{equation}

Согласно действующему законодательству отчисления в фонд социальной защиты населения составляют \num{\socialProtectionNormative\%} , в фонд обязательного страхования "--- \num{\socialNeedsNormative\%}, от фонда основной и дополнительной заработной платы исполнителей.
Общие отчисления на социальную защиту рассчитываются по формуле
\begin{equation}
  \label{eq:econ:soc_prot}
  \text{З}_{\text{сз}} = 
    \frac{(\text{З}_{\text{о}} + \text{З}_{\text{д}}) \cdot \text{Н}_{\text{сз}}}
         {\num{100\%}}
\end{equation}

Подставив вычисленные ранее значения в формулу~(\ref{eq:econ:soc_prot}) получаем
\begin{equation}
  \label{eq:econ:soc_prot_calc}
  \text{З}_{\text{сз}} =
    \frac{ (\num{\totalSalary} + \num{\additionalSalary}) \times \num{\socialProtectionFund\%} }
         { \num{100\%} }
    \approx \SI{\socialProtectionCost}{\text{\byr}}
\end{equation}

По статье <<материалы>> проходят расходы на магнитные носители информации, бумагу, красящие ленты и другие материалы, используемые при разработке ПО.
Норма расходов $ \text{Н}_{\text{мз}} $ определяется либо в расчете на \num{100} строк исходного кода, либо в процентах к основной зарплате исполнителей \mbox{\num{3\%}\,---\,\num{5\%}}.
В соответствии со справочными материалами~\cite[с.\,69, приложение~5]{econ_palicyn},
норма расхода для разрабатываемого ПО составляет $ \text{Н}_\text{м} $ = $0,38 + 0,46 + 0,43$ = \SI{\stuffOnHundredLoC}{\text{\byr} / \text{100 LoC}}
Затраты на материалы вычисляются по формуле
\begin{equation}
  \label{eq:econ:stuff}
  \text{М}_i = 
    \text{Н}_\text{м}
    \frac{ \text{V}_{\text{оi}} }
         { \num{100} } =
    \num{\stuffOnHundredLoC}
    \frac{ \num{\totalProgramSizeCorrected} }
         { \num{100} } \approx
    \SI{\stuffCost}{\text{\byr}}
\end{equation}

Расходы по статье <<машинное время>> включают оплату машинного времени, необходимого для разработки и отладки ПО, которое определяется по нормативам в машино-часах на \num{100} строк исходного кода в зависимости от характера решаемых задач и типа ПК, и вычисляются по формуле
\begin{equation}
  \label{eq:econ:machine_time}
  \text{Р}_{\text{м}} =
    \text{Ц}_{\text{м}} \cdot 
    \frac {\text{V}_{\text{о}}}
          {\num{100}} \cdot
    \text{Н}_{\text{мв}} \text{\,,}
\end{equation}
\begin{explanation}
  где & $ \text{Ц}_{\text{м}} $ & цена одного часа машинного времени, \byr; \\
      & $ \text{Н}_{\text{мв}} $ & норматив расхода машинного времени на отладку 100 строк исходного кода, часов.
\end{explanation}

Согласно справочным данным~\cite[с.\,69, приложениe~6]{econ_palicyn} норматив расхода машинного времени для разрабатываемого ПС на отладку \num{100} строк исходного кода с учетом понижающего коэффициента применения ПО для отладки \num{\reducingTimeToDebugFactor} составляет $ \text{Н}_{\text{мв}} = 0,3 \times (12 + 15) = \num{\adjustedTimeToDebugCodeNormative} $.
Цена одного часа машинного времени составляет $ \text{Ц}_{\text{м}} = \SI{\oneHourMachineTimeCost}{\text{\byr}} $.
Подставляя известные данные в формулу~(\ref{eq:econ:machine_time}) получаем
\begin{equation}
  \label{eq:econ:machine_time_calc}
  \text{Р}_{\text{м}} =
    \num{\oneHourMachineTimeCost} \times 
    \frac {\num{\totalProgramSizeCorrected}}
          {\num{100}} \times
    \num{\adjustedTimeToDebugCodeNormative} =
    \SI{\machineTimeCost}{\text{\byr}}
\end{equation}

Расходы по статье <<научные командировки>> вычисляются по нормативу как процент от основной заработной платы.
Вычисления производятся по формуле
\begin{equation}
  \label{eq:econ:business_trip}
  \text{Р}_{\text{к}} =
    \frac{ \text{З}_{\text{о}} \cdot \text{Н}_{\text{к}} }
         { \num{100\%} }
\end{equation}
\begin{explanation}
  где & $ \text{Н}_{\text{к}} $ & норматив командировочных расходов по отношению к основной заработной плате, $ \% $.
\end{explanation}

Подставляя ранее вычисленные значения в формулу~(\ref{eq:econ:business_trip}) и приняв значение $ \text{Н}_{\text{к}} = \num{\businessTripNormative\%} $ получаем
\begin{equation}
  \label{eq:econ:business_trip_calc}
    \text{Р}_{\text{к}} =
    \frac{ \num{\totalSalary} \times \num{\businessTripNormative\%} }
         { \num{100\%} } = \SI{\businessTripCost}{\text{\byr}}
\end{equation}

Статья расходов <<прочие затраты>> включает в себя расходы на приобретение и подготовку специальной научно-технической информации и специальной литературы.
Затраты определяются по нормативу принятому в организации в процентах от основной заработной платы и вычисляются по формуле
\begin{equation}
  \label{eq:econ:other_cost}
  \text{П}_{\text{з}} =
    \frac{ \text{З}_{\text{о}} \cdot \text{Н}_{\text{пз}} }
         { \num{100\%} }
\end{equation}
\begin{explanation}
  где & $ \text{Н}_{\text{пз}} $ & норматив прочих затрат в целом по организации, $ \% $.
\end{explanation}

Приняв значение норматива прочих затрат $ \text{Н}_{\text{пз}} = \num{\otherCostNormative\%} $ и подставив вычисленные ранее значения в формулу~(\ref{eq:econ:other_cost}) получаем
\begin{equation}
  \label{eq:econ:other_cost_calc}
  \text{П}_{\text{з}} =
    \frac{ \num{\totalSalary} \times \num{\otherCostNormative\%} }
         { \num{100\%} } = 
    \SI{\otherCost}{\text{\byr}}
\end{equation}

Статья <<накладные расходы>> учитывает расходы, необходимые для содержания аппарата управления, вспомогательных хозяйств и опытных производств, а также расходы на общехозяйственные нужны. Данная статья затрат рассчитывается по нормативу от основной заработной платы и вычисляется по формуле.

\begin{equation}
  \label{eq:econ:overhead_cost}
  \text{Р}_{\text{н}} =
    \frac{ \text{З}_{\text{о}} \cdot \text{Н}_{\text{рн}} }
         { \num{100\%} }
\end{equation}
\begin{explanation}
  где & $ \text{Н}_{\text{рн}} $ & норматив накладных расходов в организации,~$ \% $.
\end{explanation}

Приняв норму накладных расходов $ \text{Н}_{\text{рн}} = \num{\overheadCostNormative\%} $ и подставив известные данные в формулу~(\ref{eq:econ:overhead_cost}) получаем
\begin{equation}
  \label{eq:econ:overhead_cost_calc}
  \text{Р}_{\text{н}} =
    \frac{\num{\totalSalary} \times \num{\overheadCostNormative\%}}{\num{100\%}} = 
    \SI{\overheadCost}{\text{\byr}}
\end{equation}

Общая сумма расходов по смете на ПО рассчитывается по формуле
\begin{equation}
  \label{eq:econ:overall_cost}
  \text{С}_{\text{р}} =
    \text{З}_{\text{о}} +
    \text{З}_{\text{д}} +
    \text{З}_{\text{сз}} +
    %\text{Н}_{\text{е}} +
    \text{М} +
    % \text{Р}_{\text{с}} + % спецоборудование не нужно
    \text{Р}_{\text{м}} +
    \text{Р}_{\text{нк}} +
    \text{П}_{\text{з}} +
    \text{Р}_{\text{н}}
\end{equation}

Подставляя ранее вычисленные значения в формулу~(\ref{eq:econ:overall_cost}) получаем

\begin{equation}
  \label{eq:econ:overall_cost_calc}
  \text{С}_{\text{р}} = \SI{\overallCost}{\text{\byr}}
\end{equation}

% Расходы на сопровождение и адаптацию, которые несет производитель ПО, вычисляются по нормативу от суммы расходов по смете и рассчитываются по формуле
% \begin{equation}
%   \label{eq:econ:software_support}
%   \text{Р}_{\text{са}} = 
%     \frac { \text{С}_{\text{р}} \cdot \text{Н}_{\text{рса}} }
%           { \num{100\%} } \text{\,,}
% \end{equation}
% \begin{explanation}
%   где & $ \text{Н}_{\text{рса}} $ & норматив расходов на сопровождение и адаптацию ПО,~$ \% $.
% \end{explanation}

% Приняв значение норматива расходов на сопровождение и адаптацию $ \text{Н}_{\text{рса}} = \num{\supportNormative\%} $ и подставив ранее вычисленные значения в формулу~(\ref{eq:econ:software_support}) получаем
% \begin{equation}
%   \label{eq:econ:software_support_calc}
%   \text{Р}_{\text{са}} = 
%     \frac { \num{\overallCost} \times \num{\supportNormative\%} }
%           { \num{100\%} } \approx \SI{\softwareSupportCost}{\text{\byr}} \text{\,.}
% \end{equation}

% Полная себестоимость создания ПО включает сумму затрат на разработку, сопровождение и адаптацию и вычисляется по формуле
% \begin{equation}
%   \label{eq:econ:base_cost}
%   \text{С}_{\text{п}} = \text{С}_{\text{р}} + \text{Р}_{\text{са}} \text{\,.}
% \end{equation}

% Подставляя известные значения в формулу~(\ref{eq:econ:base_cost}) получаем
% \begin{equation}
%   \label{eq:econ:base_cost_calc}
%   \text{С}_{\text{п}} = \num{\overallCost} + \num{\softwareSupportCost} = \SI{\baseCost}{\text{\byr}} \text{\,.}
% \end{equation}



\subsection{Расчёт экономической эффективности у разработчика}

Важная задача при выборе проекта для финансирования это расчет экономической эффективности проектов и выбор наиболее выгодного проекта.
На основании анализа рыночных условий и договоренности с заказчиком об отпускной цене прогнозируемая рентабельность проекта составит~$ \text{У}_{\text{рп}} = \num{\profitability\%} $.
Прибыль рассчитывается по формуле
\begin{equation}
  \label{eq:econ:income}
  \text{П}_{\text{с}} = 
    \frac { \text{С}_{\text{п}} \cdot \text{У}_{\text{рп}} }
          { \num{100\%} }
\end{equation}
\begin{explanation}
  где & $ \text{П}_{\text{с}} $ & прибыль от реализации ПО заказчику, \byr; \\
      & $ \text{У}_{\text{рп}} $ & уровень рентабельности ПО,~$ \% $.
\end{explanation}

Подставив известные данные в формулу~(\ref{eq:econ:income}) получаем прогнозируемую прибыль от реализации ПО
\begin{equation}
  \label{eq:econ:income_calc}
  \text{П}_{\text{с}} = 
    \frac { \num{\baseCost} \times \num{\profitability\%} }
          { \num{100\%} } 
    \approx \SI{\income}{\text{\byr}}
\end{equation}

Прогнозируемая цена ПО без учета налогов включаемых в цену вычисляется по формуле 
\begin{equation}
  \label{eq:econ:estimated_price}
  \text{Ц}_{\text{п}} = \text{С}_{\text{п}} + \text{П}_{\text{с}}
\end{equation}

Подставив данные в формулу~(\ref{eq:econ:estimated_price}) получаем цену ПО без налогов
\begin{equation}
  \label{eq:econ:estimated_price_calc}
  \text{Ц}_{\text{п}} = \num{\baseCost}  + \num{\income} = \SI{\estimatedPrice}{\text{\byr}}
\end{equation}

% Отчисления и налоги в местный и республиканский бюджеты вычисляются по формуле
% \begin{equation}
%   \label{eq:econ:local_repub_tax}
%   \text{О}_{\text{мр}} =
%     \frac { \text{Ц}_{\text{п}} \cdot \text{Н}_{\text{мр}} }
%           { \num{100\%} - \text{Н}_{\text{мр}} } \text{\,,}
% \end{equation}
% \begin{explanation}
%   где & $ \text{Н}_{\text{мр}} $ & норматив отчислений в местный и республиканский бюджеты, \byr.
% \end{explanation}

% Приняв норматив отчислений в местный и республиканский бюджеты $ \text{Н}_{\text{мр}} = \num{\localRepubTaxNormative\%} $ и подставив известные данные в формулу~(\ref{eq:econ:local_repub_tax}) получим величину единого платежа
% \begin{equation}
%   \label{eq:econ:local_repub_tax_calc}
%   \text{О}_{\text{мр}} = 
%     \frac { \num{\estimatedPrice} \cdot \num{\localRepubTaxNormative\%} }
%           { \num{100\%} - \num{\localRepubTaxNormative\%} } 
%     \approx \SI{\localRepubTax}{\text{\byr}}
% \end{equation}

Налог на добавленную стоимость рассчитывается по формуле
\begin{equation}
  \label{eq:econ:nds}
  \text{НДС}_{\text{}} =
    \frac{ \text{Ц}_{\text{п}} \cdot \text{Н}_{\text{дс}} }
         { \num{100\%} }
\end{equation}
\begin{explanation}
  где & $ \text{Н}_{\text{дс}} $ & норматив НДС,~$\%$.
\end{explanation}

Норматив НДС составляет $ \text{Н}_{\text{дс}} = \num{\ndsNormative\%} $, подставляя известные значения в формулу~(\ref{eq:econ:nds}) получаем
\begin{equation}
  \text{НДС} =
  \frac { \num{\estimatedPrice} \times \num{\ndsNormative\%} }
          { \num{100\% }} 
    \approx \SI{\nds}{\text{\byr}}
\end{equation}

Расчет прогнозируемой отпускной цены осуществляется по формуле 
\begin{equation}
  \label{eq:econ:selling_price}
  \text{Ц}_{\text{о}} = \text{Ц}_{\text{п}} + \text{НДС}
\end{equation}

Подставляя известные данные в формулу~(\ref{eq:econ:selling_price}) получаем прогнозируемую отпускную цену
\begin{equation}
  \label{eq:econ:selling_price_calc}
  \text{Ц}_{\text{о}} = \num{\estimatedPrice} + \num{\nds} \approx \SI{\sellingPrice}{\text{\byr}}
\end{equation}


Чистую прибыль от реализации проекта можно рассчитать по формуле
\begin{equation}
  \label{eq:econ:income_with_taxes}
  \text{П}_\text{ч} = 
    \text{П}_\text{c} \cdot
    \left(1 - \frac{ \text{Н}_\text{п} }{ \num{100\%} } \right)
\end{equation}
\begin{explanation}
  где & $ \text{Н}_{\text{п}} $ & величина налога на прибыль,~$\%$.
\end{explanation}

Приняв значение налога на прибыль $ \text{Н}_{\text{н}} = \num{\taxForIncome\%} $ и подставив известные данные в формулу~(\ref{eq:econ:income_with_taxes}) получаем чистую прибыль
\begin{equation}
  \label{eq:econ:income_with_taxes_calc}
  \text{П}_\text{ч} = 
    \num{\income} \times \left( 1 - \frac{\num{\taxForIncome\%}}{100\%} \right) = \SI{\incomeWithTaxes}{\text{\byr}} \text{\,.}
\end{equation}

В качестве экономического эффекта для разработчика можно рассматривать чистую прибыль от реализации~$ \text{П}_\text{ч} $.
Рассчитанные данные приведены в таблице~\ref{table:econ:calculated_data}.

\begin{table}[!h!t]
\caption{Рассчитанные данные}
\label{table:econ:calculated_data}
  \centering
  \begin{tabular}{| >{\raggedright}m{0.55\textwidth} 
                  | >{\centering}m{0.17\textwidth} 
                  | >{\centering\arraybackslash}m{0.20\textwidth}|}
    \hline
    {\begin{center}
      Наименование
    \end{center} } & Условное обозначение & Значение \\
    \hline
    Нормативная трудоемкость, чел.$/$дн. & $ \text{Т}_\text{н} $ & \num{\normativeManDays} \\

    \hline
    Общая трудоемкость разработки, чел.$/$дн. & $ \text{Т}_\text{о} $ & \num{\adjustedManDays} \\

    \hline
    Численность исполнителей, чел. & $ \text{Ч}_\text{р} $ & \num{\requiredNumberOfProgrammers} \\

    \hline
    Часовая тарифная ставка программиста \Rmnum{1}-разряда, \byr{}$/$ч. & $ \text{Т}_{\text{ч}}^{\text{прогр. \Rmnum{1}-разр.}} $ & \num{\salaryPerHourFst} \\

    \hline
    Часовая тарифная ставка ведущего программиста, \byr{}$/$ч. & $ \text{Т}_{\text{ч}}^{\text{вед. прогр.}} $ & \num{\salaryPerHourSnd} \\

    \hline
    Основная заработная плата, \byr{} & $ \text{З}_\text{о} $ & \num{\totalSalary} \\

    \hline
    Дополнительная заработная плата, \byr{} & $ \text{З}_\text{д}$ & \num{\additionalSalary} \\

    \hline
    Отчисления в фонд социальной защиты, \byr{} & $ \text{З}_\text{сз} $ & \num{\socialProtectionCost} \\

    \hline
    Затраты на материалы, \byr{} & $ \text{М} $ & \num{\stuffCost} \\

    \hline
    Расходы на машинное время, \byr{} & $ \text{Р}_\text{м} $ & \num{\machineTimeCost} \\

    \hline
    Расходы на командировки, \byr{} & $ \text{Р}_\text{к} $ & \num{\businessTripCost} \\

    \hline
    Прочие затраты, \byr{} & $ \text{П}_\text{з} $ & \num{\otherCost} \\

    \hline
    Накладные расходы, \byr{} & $ \text{Р}_\text{н} $ & \num{\overheadCost} \\

    % \hline
    % Общая сумма расходов по смете, \byr{} & $ \text{С}_\text{р} $ & \num{\overallCost} \\

    % \hline
    % Расходы на сопровождение и адаптацию, \byr{} & $ \text{Р}_\text{са} $ & \num{\softwareSupportCost} \\

    \hline
    Полная себестоимость, \byr{} & $ \text{С}_\text{п} $ & \num{\baseCost} \\

    \hline
    Прогнозируемая прибыль, \byr{} & $ \text{П}_\text{с} $ & \num{\income} \\

    \hline
    НДС, \byr{} & $ \text{НДС} $ & \num{\nds} \\

    \hline
    Прогнозируемая отпускная цена ПО, \byr{} & $ \text{Ц}_\text{о} $ & \num{\sellingPrice} \\

    \hline
    Чистая прибыль, \byr{} & $ \text{П}_\text{ч} $ & \num{\incomeWithTaxes} \\

    \hline
  \end{tabular}
\end{table}
\hfill
\clearpage

\subsection{Расчет экономии основных видов ресурсов в связи с использованием нового ПО}

\FPeval{\avgProgrammerSalary}{3000000}
\FPeval{\avgWorkBefore}{80}
\FPeval{\avgWorkAfter}{60}
\FPeval{\clientWorkingHoursInDay}{8}
\FPeval{\avgWorkingDaysInMonth}{21.5}
\FPeval{\tasksInYear}{100}

\FPeval{\salaryEconTaskExact}{ (\avgProgrammerSalary * (\avgWorkBefore - \avgWorkAfter)) / (\clientWorkingHoursInDay * \avgWorkingDaysInMonth) }
\FPround{\salaryEconTask}{\salaryEconTaskExact}{2}

\FPeval{\salaryEconExact}{ \salaryEconTask * \tasksInYear }
\FPround{\salaryEcon}{\salaryEconExact}{2}

\FPeval{\salaryEconAddExact}{ \salaryEcon * 1.5 }
\FPround{\salaryEconAdd}{\salaryEconAddExact}{2}

\FPeval{\econTotal}{ \salaryEconAdd }

\FPeval{\deltaAddIncomeExact}{ \econTotal - \econTotal * \taxForIncome / 100.0 }
\FPround{\deltaAddIncome}{\deltaAddIncomeExact}{2}

При расчёте затрат на заработную плату важно правильно выбрать показатели сравниваемых проектов в зависимости от особенностей сервиса.
В качестве показателей для сравнения базового варианта и проектируемого ПО можно использовать:
\begin{itemize}
  \item трудоёмкость решаемых задач (на основе хронометражных наблюдений и накопленной статистической информации);
  \item затраты труда на 100 строк исходного кода;
  \item количество выполнения транзакций в минуту;
  \item потери времени из-за простоев сервиса.
\end{itemize}

При выполнении технико-экономического обоснования в качестве показателя для сравнения была выбрана трудоёмкость решаемых задач.
Для определение величины экономического эффекта определим экономию затрат на заработную плату в расчете на одну задачу

\begin{equation}
  \label{eq:econ:salary_econ_task}
  \text{С}_\text{зе} = 
    \frac{\text{З}_\text{см} \cdot (\text{Т}_\text{с1} - \text{Т}_\text{с2})}
         {\text{Т}_\text{ч} \text{Д}_\text{р}}
\end{equation}
\begin{explanation}
  где & $ \text{З}_\text{см} $ & среднемесячная заработная плата одного программиста,~$\byr$; \\
      & $ \text{Т}_\text{с1} - \text{Т}_\text{с2} $ & снижение трудоемкости работ в расчете на одну задачу,~$\text{чел"=ч}$; \\
      & $ \text{Т}_\text{ч} $ & количество часов работы в день; \\
      & $ \text{Д}_\text{р} $ & среднемесячное количество рабочих дней.
\end{explanation}

Общая экономия затрат на заработную плату в год определяется как

\begin{equation}
  \label{eq:econ:salary_econ}
  \text{С}_\text{з} = \text{С}_\text{зе} \cdot \text{А}_\text{2}
\end{equation}
\begin{explanation}
  где & $ \text{А}_\text{2} $ & количество типовых задач, решаемых за год.
\end{explanation}

На основе данных, предоставленных пользователем, принимая $ \text{З}_\text{см} $ = \SI{\avgProgrammerSalary}{\byr},
$ \text{Т}_\text{с1} $ = \SI{\avgWorkBefore}{\text{чел"=ч}}, $ \text{Т}_\text{с2} $ = \SI{\avgWorkAfter}{\text{чел"=ч}},
$ \text{Т}_\text{Ч} $ = \SI{\clientWorkingHoursInDay}{\text{ч}}, $ \text{Д}_\text{р} $ = \SI{\avgWorkingDaysInMonth}{\text{дн.}},
$ \text{А}_\text{2} $ = \SI{\tasksInYear}{\text{шт}}, получаем

\begin{equation}
  \label{eq:econ:salary_econ_task_calc}
  \text{С}_\text{зе} = 
    \frac{\num{\avgProgrammerSalary} \cdot (\num{\avgWorkBefore} - \num{\avgWorkAfter})}
         {\num{\clientWorkingHoursInDay} \cdot \num{\avgWorkingDaysInMonth}} \approx
         \SI{\salaryEconTask}{\byr{}}
\end{equation}

\begin{equation}
  \label{eq:econ:salary_econ_calc}
  \text{С}_\text{з} = \num{\salaryEconTask} \cdot \num{\tasksInYear} =
  \SI{\salaryEcon}{\byr{}}
\end{equation}

С учетом начисления на зарплату:

\begin{equation}
  \label{eq:econ:salary_econ_add_calc}
  \text{С}_\text{н} = \num{\salaryEcon} \cdot 1.5 =
  \SI{\salaryEconAdd}{\byr{}}
\end{equation}

В нашем случае общая годовая экономия текущих затрат состоит только из экономии на заработной плате: $\text{С}_\text{о}$ = $\text{С}_\text{н}$ = \SI{\econTotal}{\byr{}}.

Внедрение нового ПО позволит пользователю сэкономить на текущих затратах, т.е. практически получить на эту сумму дополнительную прибыль.
Для пользователя в качестве экономического эффекта выступает чистая прибыль "--- дополнительная прибыль, остающаяся в его распоряжении, которая определяется по формуле

\begin{equation}
  \label{eq:econ:additional_income}
  \Delta \text{П}_\text{ч} = 
    \text{С}_\text{о} - 
    \frac{\text{С}_\text{о} \cdot \text{Н}_\text{п}}
         {\num{100\%}}
\end{equation}
\begin{explanation}
  где & $ \text{Н}_\text{п} $ & ставка налога на прибыль, $\%$.
\end{explanation}

Подставляя известные данные в~\ref{eq:econ:additional_income}, получим

\begin{equation}
  \label{eq:econ:additional_income_calc}
  \Delta \text{П}_\text{ч} = 
    \num{\econTotal} -
    \frac{\num{\econTotal} \cdot \num{\taxForIncome}}
         {\num{100\%}} \approx
    \SI{\deltaAddIncome}{\byr}
\end{equation}

Полученные суммы результатов (прибыли) и затрат (капитальных вложений) по годам приводят к единому времени "--- расчетному году.
Для оценки изменения стоимости денег необходимо оценить норму дисконта, которая вычисляется по формуле

\begin{equation}
  \label{eq:econ:discount_rate}
  \text{Е} = r + s + \sum_{i=1}^{n} g_i \text{\,,}
\end{equation}
\begin{explanation}
где & $ r $ & реальная (без учета компенсации за инфляцию) безрисковая ставка ссудного процента, \%; \\
    & $ s $ & инфляционное ожидание за период $ t $, рассчитанное как среднее за расчетный период проекта, \%; \\
    & $ g_i $ & премия за отдельный риск по конкретному фактору, \%.
\end{explanation}

По состоянию на 2015 год безрисковая ставка ссудного процента составляет $r = \num{25\%}$, возможное влияние непредвиденных обстоятельств на величину этой ставки оценим премией за риск $ g_1 = \num{1\%} $.
По различным данным инфляция в 2015 году в Беларуси составит $ s = \num{15\%} $. Дополнительно учтем риски падения спроса $ g_2 = \num{1\%} $ и падения дохода $ g_3 = \num{1\%}$.
Подставляя приведенные данные в формулу~(\ref{eq:econ:discount_rate}) получаем
\begin{equation}
  \label{eq:econ:discount_rate_calc}
  \text{Е} = \num{25\%} + \num{15\%} + \num{1\%} + \num{1\%} + \num{1\%} = \num{43\%} \text{\,.}
\end{equation}

Зная норму дисконта можно рассчитать значения коэффициентов дисконтирования для каждого из отрезков времени расчетного периода по формуле~(\ref{eq:econ:discount_factor}).
\begin{equation}
  \label{eq:econ:discount_factor}
  \alpha_t = (1 + \text{Е})^{-t}
\end{equation}
\begin{explanation}
где & $ t $ & номер отрезка времени расчетного периода с момента старта проекта, \%.
\end{explanation}

Таким образом, значения коэффициентов дисконтирования для последующих 3 лет с момента старта проекта составляют:

\begin{equation}
  \label{eq:econ:discount_factor}
  \begin{aligned}
    \alpha^{2015}_t &= (1 + 0,43)^{0} = 1 \\
    \alpha^{2016}_t &= (1 + 0,43)^{1} \approx 0,70 \\
    \alpha^{2017}_t &= (1 + 0,43)^{2} \approx 0,49 \\
    \alpha^{2018}_t &= (1 + 0,43)^{3} \approx 0,34
  \end{aligned}
\end{equation}

\FPeval{\buyCost}{90000000}
\FPeval{\teachCost}{1000000}
\FPeval{\supportCost}{2000000}
\FPeval{\totalCost}{clip( \buyCost + \teachCost + \supportCost )}

\FPround{\deltaAddIncomeRound}{\deltaAddIncome}{0}
\FPeval{\deltaAddIncomeOneExact}{0.7 * \deltaAddIncome}
\FPeval{\deltaAddIncomeTwoExact}{0.49 * \deltaAddIncome}
\FPeval{\deltaAddIncomeThreeExact}{0.34 * \deltaAddIncome}
\FPround{\deltaAddIncomeOne}{\deltaAddIncomeOneExact}{0}
\FPround{\deltaAddIncomeTwo}{\deltaAddIncomeTwoExact}{0}
\FPround{\deltaAddIncomeThree}{\deltaAddIncomeThreeExact}{0}

\FPeval{\resultCostDiffZero}{clip(\deltaAddIncomeRound - \totalCost)}
\FPeval{\resultCostDiffOne}{\deltaAddIncomeOne}
\FPeval{\resultCostDiffTwo}{\deltaAddIncomeTwo}
\FPeval{\resultCostDiffThree}{\deltaAddIncomeThree}

\FPeval{\resultCostAccumDiffZero}{clip(0 + \resultCostDiffZero)}
\FPeval{\resultCostAccumDiffOne}{clip(0 + \resultCostDiffZero + \resultCostDiffOne)}
\FPeval{\resultCostAccumDiffTwo}{clip(0 + \resultCostDiffZero + \resultCostDiffOne + \resultCostDiffTwo)}
\FPeval{\resultCostAccumDiffThree}{clip(0 + \resultCostDiffZero + \resultCostDiffOne + \resultCostDiffTwo + \resultCostDiffThree)}

Также для определения экономической целесообразности нам понадобятся данные о расходах на стороне покупателя.
Принимем затраты на приобретение ПО $ \text{К}_\text{пр} $ = \SI{\buyCost}{\byr},
на освоение ПО $ \text{К}_\text{ос} $ = \SI{\teachCost}{\byr},
на сопровождение ПО $ \text{К}_\text{с} $ = \SI{\supportCost}{\byr}.

Все рассчитанные данные экономического эффекта сводятся в таблицу (табл.~\ref{table:econ:customer_econ_effect}).

\begin{table}[ht]
\caption{Расчет экономического эффекта от использования нового ПО}
\label{table:econ:customer_econ_effect}
\centering
  \begin{tabular}{| >{\raggedright}m{0.31\textwidth}
                  | >{\centering}m{0.14\textwidth}
                  | >{\centering}m{0.14\textwidth}
                  | >{\centering}m{0.14\textwidth}
                  | >{\centering\arraybackslash}m{0.14\textwidth}|}

  \hline
       \multirow{2}{0.31\textwidth}{\centering Показатели}
       & \multicolumn{4}{c|}{\centering Годы} \tabularnewline

  \cline{2-5}
  & { 2015 } & { 2016 } & { 2017 } & { 2018 } \tabularnewline

  \hline Результаты: & & & & \tabularnewline
  \hline Прирост прибыли за счет экономии затрат ($\text{П}_\text{ч}$): & \deltaAddIncomeRound & \deltaAddIncomeRound & \deltaAddIncomeRound & \deltaAddIncomeRound \tabularnewline
  \hline То же самое с учетом фактора времени: & \deltaAddIncomeRound & \deltaAddIncomeOne & \deltaAddIncomeTwo & \deltaAddIncomeThree \tabularnewline

  \hline Затраты: & & & & \tabularnewline
  \hline Приобретение ПО ($\text{К}_\text{пр}$): & \buyCost & & & \tabularnewline
  \hline Освоение ПО ($\text{К}_\text{ос}$): & \teachCost & & & \tabularnewline
  \hline Сопровождение ПО ($\text{К}_\text{с}$): & \supportCost & & & \tabularnewline
  \hline Всего затрат: & \totalCost & & & \tabularnewline

  \hline Экономический эффект: & & & & \tabularnewline
  \hline Превышение результата над затратами: & \resultCostDiffZero & \resultCostDiffOne & \resultCostDiffTwo & \resultCostDiffThree \tabularnewline
  \hline То же с нарастающим итогом: & \resultCostAccumDiffZero & \resultCostAccumDiffOne & \resultCostAccumDiffTwo & \resultCostAccumDiffThree \tabularnewline
  \hline
  \end{tabular}
\end{table}
\hfill
\clearpage

\subsection{Выводы}

По результатам выполненного технико-экономического обоснования можно сказать,
что разрабатываемое ПС определения вероятных мест скопления людей в помещениях является экономически целесообразным.

Положительный экономический эффект достигается за счет экономии затрат на заработную плату сотрудников путем уменьшения трудоемкости работ.

Чистая прибыль от реализации разрабатываемого ПС составляет $ \text{П}_\text{ч} $ = \SI{\incomeWithTaxes}{\byr{}}
и представляет собой экономический эффект разработчика от создания проектируемого ПС.

Прогнозируемая отпускная цена составляет $ \text{Ц}_\text{о} $ = \SI{\sellingPrice}{\byr{}}.

На стороне заказчика проект окупится уже через 2 года после введения в эксплуатацию,
а через 3 года общий дисконтированный экономический эффект в виде превышения прироста прибыли над затратами составит \SI{\resultCostAccumDiffThree}{\byr}.

