\sectioncentered*{Реферат}
\thispagestyle{empty}

\emph{Ключевые слова}: моделирование пешеходных потоков; места скопления людей; модель социальных сил; карта плотности пешеходных потоков.

\vspace{4\parsep}

Дипломный проект выполнен на 6 листах формата А1 с пояснительной запиской на~\pageref*{LastPage} страницах, с приложением, содержащим исходные коды разработанного программного средства.
Пояснительная записка включает 7~глав, \totfig{}~рисунков, \tottab{}~таблиц, \toteq{}~формулы и \totref{}~литературный источник.

Целью дипломного проекта является разработка ПС определения вероятных мест скопления людей в помещениях,
направленного на решение актуальных задач, возникающих при разработке архитектурных проектов.

Для достижения цели дипломного проекта было разработано ПС, определяющее вероятные места скопления людей в помещениях.
Данное ПС может быть использовано при разработке реальных архитектурных проектов.

В разделе технико"=экономического обоснования был произведен расчет затрат на создание ПО, а также срока окупаемости данной разработки.
Проведенные расчеты показали экономическую целесообразность проекта.

Пояснительная записка включает раздел по охране труда, в котором была произведена оценка безопасности условий труда при разработке данного дипломного проекта.

\clearpage
