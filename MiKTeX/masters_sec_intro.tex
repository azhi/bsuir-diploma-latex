\sectioncentered*{Введение}
\addcontentsline{toc}{section}{Введение}

\label{sec:intro}

Темой данной работы является корректировка разработанной в рамках дипломного проекта~\cite{my_diploma} имитационной модели пешеходных потоков для работы в условиях массовой паники.

Упомянутый дипломный проект ставил своей целью разработку ПС, способного предсказывать вероятные места скопления людей в помещениях.
Предполагаемые задачи, решаемые данным ПС, включали в себя задачу по оценке безопасности исследуемого сооружения.
Данная работа значительно расширяет возможности разработанной имитационной модели и ПС, позволяя сконцентрироваться на исследовании таких важных параметров сооружения, как время эвакуации.

Однако сценарий эвакуации отличается от нормального функционирования сооружения "--- при эвакуации возможно возникновение массовой паники.
Именно поэтому данная работа ставит своей целью модификацию модели для работы в условиях массовой паники.
С помощью модифицированного ПС можно будет оценить время эвакуации из заданного сооружения при различных "уровнях" паники.
Подробнее разработанная модель и ее характеристики будут описаны во втором разделе данной работы под названием "Моделирование паники".
В первом же разделе ("Обзор существующих моделей паники") будет произведен подробный обзор существующей научной литературы на обсуждаемую тему.

В третьем разделе будет описан процесс проектирования модификаций ПС.
В данном разделе будет множество ссылок на разработанный дипломный проект как на основу данной работы, а также описание конкретных изменений, которые были внесены в архитектуру и реализацию ПС.
В конце раздела будет произведена оценка решений, принятых при разработке базового дипломного проекта - насколько каждое из них помогло либо осложнило разработку модификации.

В четвертом разделе под названием <<Методика использования разработанного программного средства>> будет приведено измененное руководство пользователя к разрабатываемому ПС "--- описаны все изменения, внесенные в пользовательский интерфейс и файлы конфигурации.

Пятый раздел будет посвящен оценке результатов, полученных с помощью ПС.
И наконец в заключении будут подведены итоги работы и отмечены недостатки разработанного ПС.
